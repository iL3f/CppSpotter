%%%%%%%%%%%%%%%%%%%%%%%%%%%%%%%%%%%%%%%%%%%%%%%%%%%%%%%%%%%%%%%%%%%%%%%%%%%%%%%%
\intro
%%%%%%%%%%%%%%%%%%%%%%%%%%%%%%%%%%%%%%%%%%%%%%%%%%%%%%%%%%%%%%%%%%%%%%%%%%%%%%%%
В настоящее время компьютеры уже проникли во все сферы жизни человека. С каждым годом все больше
возрастает сложность программного обеспечения, что влечет за собой увеличение количества программных 
ошибок. Некоторые программные ошибки незначительно влияют на работу программы и могут быть
не обнаружены длительное время, однако другие могут привести к некорректной работе, зависанию или
завершению программы. Известны случаи, когда ошибки в программе привели к серьезным последствиям.
Например ошибка в ПО аппарата для лучевой терапии Therac-25 стала причиной смерти 
нескольких пациентов \cite{therac}, ошибка в системе управления полетом ракеты привела к падению
и стала самой дорогой в истории\cite{ariane}. По подсчетам экономистов программные ошибки обходятся в миллиарды долларов в год. 

Очевидно, что обеспечение качества ПО является одной из важнейших задач в сфере информационных технологий.
Для обеспечения качества могут применяться такие методы как:
\begin{itemize}
	\item Тестирование
	\item Статический анализ кода
	\item Динамический анализ кода
	\item Инспекция кода
\end{itemize}

В данной работе будет рассмотрен подход использования статического анализа исходного кода.
Для этого был разработан модуль расширения статических проверок для компилятора Clang. Разработанный 
модуль расширения представляет собой динамическую библиотеку, которая подключается к компилятору 
Clang. Модуль расширения занимает небольшой объем памяти и для проведения статического анализа 
пользователю не нужно устанавливать специальные программы, достаточно добавить флаги для компилятора, 
подключающие разработанный модуль. 

Для выявления фрагментов кода, потенциально содержащих ошибки, используется библиотека AST Matchers.
Данная библиотека предоставляет краткий способ определения шаблонов по которым происходит 
поиск фрагментов кода в абстрактном синтаксическом дереве Clang. В результате нет необходимости писать 
много повторяющегося код для похожих шаблонов.

Разработчики статического анализатора PVS-Studio постояно производят поиск ошибок в открытых проектах.
Из проведенных анализов видно, что часто ошибки допускают по невнимательности.
Поэтому при выборе проверок для выявления ошибок, были выбраны ошибки появляющиеся во время дублирования 
существующего исходного кода, а так же ошибки, которые допускают программисты по невнимательности.
К примеру распространенной ошибкой является ошибка вида <<выражение1 оператор выражение2>>, где
выражение1 эквивалентно выражение2. 
Выбранные ошибки присутствуют в PVS-Studio, но не представлены в Clang Static Analyzer.
