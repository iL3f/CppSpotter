%%%%%%%%%%%%%%%%%%%%%%%%%%%%%%%%%%%%%%%%%%%%%%%%%%%%%%%%%%%%%%%%%%%%%%%%%%%%%%%%
\intro
%%%%%%%%%%%%%%%%%%%%%%%%%%%%%%%%%%%%%%%%%%%%%%%%%%%%%%%%%%%%%%%%%%%%%%%%%%%%%%%%
В настоящее время компьютеры уже проникли во все сфера жизни человека. С каждым годом все больше
возрастает сложность программного обеспечения, что влечет за собой увеличение количества программных 
ошибок. Некоторые программные ошибки незначительно влияют на работу программы и могут быть
не обнаружены длительное время, однако другие могут привести к некорректной работе, зависанию или
завершению программы. Известны случаи, когда ошибки в программе привели к серьезным последствиям и 
являлись причиной смерти пациентов, падением летательных аппаратов или даже могли стать причиной 
третьей мировой войны. По подсчетам экономистов программные ошибки обходятся миллиардами долларов в год. 

Очевидно, что обеспечение качества ПО является одной из важнейших задач в сфере информационных технологий.
Для оценки качества могут применяться такие методы как:
\begin{itemize}
	\item Тестирование (модульное, интеграционное, системное, регрессионное)
	\item Статический анализ кода
	\item Динамический анализ кода
	\item Обзор кода
\end{itemize}

В данной работе будет рассмотрен подход использования статического анализа исходного кода. 
Для этого был разработан модуль расширения статических проверок для компилятора Clang. Применение
статического анализа позволяет обнаружить множество программных ошибок. Статический анализ производится 
автоматезировано и требует участие человека только для рассмотрения найденных подозрительных
мест в исходном коде. Вследствие чего снижается стоимость обнаружения ошибок.  Однако, как и
у любой другой методологии нахождения ошибок, статический анализ имеет сильные и слабые стороны.
Поэтому важно понимать, что для обеспечения высокого качества ПО необходимо использовать сочетание
различных методик.