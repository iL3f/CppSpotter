%%%%%%%%%%%%%%%%%%%%%%%%%%%%%%%%%%%%%%%%%%%%%%%%%%%%%%%%%%%%%%%%%%%%%%%%%%%%%%%%
\chapter{Тестирование и анализ разработанного модуля}
%%%%%%%%%%%%%%%%%%%%%%%%%%%%%%%%%%%%%%%%%%%%%%%%%%%%%%%%%%%%%%%%%%%%%%%%%%%%%%%%

%%%%%%%%%%%%%%%%%%%%%%%%%%%%%%%%%%%%%%%%%%%%%%%%%%%%%%%%%%%%%%%%%%%%%%%%%%%%%%%%
\section{Тестирование}
%%%%%%%%%%%%%%%%%%%%%%%%%%%%%%%%%%%%%%%%%%%%%%%%%%%%%%%%%%%%%%%%%%%%%%%%%%%%%%%%
Показать нахождение ошибок. 

Список ложных срабатываний:
\begin{itemize}
	\item 
\begin{lstlisting}
*++scan == *++match && *++scan == *++match
\end{lstlisting}

	\item 
Диагностическое сообщение не выдается, если сравниваются два идентичных выражения типа float или double. 
Такое сравнение позволяет определить, является ли значение NaN. Пример кода, реализующего подобную проверку:
\begin{lstlisting}
bool isnan(double X) { return X != X; }
\end{lstlisting}

	\item
Наличие двух пустых веток считается корректной и безопасной ситуацией. Подробные конструкции 
можно часто встретить при использовании макросов. Пример безопасного кода:
\begin{lstlisting}
if (exp) {
} else {
}
\end{lstlisting}

	\item Функция foo каждый вызов меняет свое состояние или глобальные переменные 
\begin{lstlisting}
if (foo() || foo())
\end{lstlisting}
	
	\item strlen(text+1) так и задумывалось
	
	\item memset 
	
\end{itemize}

Список ненайденных ошибок:

%%%%%%%%%%%%%%%%%%%%%%%%%%%%%%%%%%%%%%%%%%%%%%%%%%%%%%%%%%%%%%%%%%%%%%%%%%%%%%%%
\section{Анализ}
%%%%%%%%%%%%%%%%%%%%%%%%%%%%%%%%%%%%%%%%%%%%%%%%%%%%%%%%%%%%%%%%%%%%%%%%%%%%%%%%

Тут будет сравнение скорости компиляции с плагином, без него и с частично отключенными проверками.