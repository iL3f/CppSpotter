%%%%%%%%%%%%%%%%%%%%%%%%%%%%%%%%%%%%%%%%%%%%%%%%%%%%%%%%%%%%%%%%%%%%%%%%%%%%%%%%
\chapter{Тестирование разработанного модуля}
%%%%%%%%%%%%%%%%%%%%%%%%%%%%%%%%%%%%%%%%%%%%%%%%%%%%%%%%%%%%%%%%%%%%%%%%%%%%%%%%

%%%%%%%%%%%%%%%%%%%%%%%%%%%%%%%%%%%%%%%%%%%%%%%%%%%%%%%%%%%%%%%%%%%%%%%%%%%%%%%%
\section{Тестирование}
%%%%%%%%%%%%%%%%%%%%%%%%%%%%%%%%%%%%%%%%%%%%%%%%%%%%%%%%%%%%%%%%%%%%%%%%%%%%%%%%
Тестирование разработанного модуля расширения производилось в два этапа:
\begin{enumerate}
	\item Тестирование на основе различных тестовых программ, содержащих и не содержащих ошибки.
	\item Тестирование на реальных проектах.
\end{enumerate}
%%%%%%%%%%%%%%%%%%%%%%%%%%
\subsection{Тестирование на основе тестовых программ}
%%%%%%%%%%%%%%%%%%%%%%%%%%

При добавлении новой проверки, создавались тестовые программы, которые содержат проверяемую ошиибку. 
После чего производился запуск процесса компиляции с подключением разработанного модуля. Если
добавленная проверка обрабатывает ошибку и выводит предупреждение, значит разработанная проверка
работает правильно. 

Таким образом в результате тестирования созданного модуля были выявленны некоторые ошибки первого
\footnote{Ошибка первого рода~--- в исходном коде ошибки нет, но происходит ложное срабатывание} и второго рода
\footnote{Ошибка второго рода~--- в исходном коде есть ошибка, но не происходит ее обнаружение}. 

\subsection*{Ошибки первого рода}
Так как анализатор выдает предупреждения, если найдены участки кода, в которых слева и справа от оператора
находятся одинаковые выражения (см. \ref{sec:eqBin}), то следующий код вызывает подозрение у анализатора: 
\begin{lstlisting}
*++scan == *++match && *++scan == *++match
\end{lstlisting}
Однако данный код является корректным, так как переменные в левой части изменяются, после чего переменные
в правой части имеют измененные значения. Так же данное ложное срабатывание будет происходить при
вызове функции, которая меняет свое состояние или глобальные переменные:
\begin{lstlisting}
bool foo()
{
	static bool initialized = false;
	if (initialized)
	{
		initialized = init(); 
		return initialized;
	}
	else
	{
		return isAvailable();
	}		
}
if (foo() && foo())
\end{lstlisting}
В приведенном выше примере вызов функции \textit{foo()} является необходимым и не соответствует ошибке, 
но анализатор не учитывает побочные эффекты (side effects) при вызове функции и выдает предупреждения.

Часто можно встретить наличие пустых веток \textit{if} и \textit{else}, которые появляются при использовании
макросов. Такая конструкция считается корректной и безопасной:
\begin{lstlisting}
if (exp) {} 
else {}
\end{lstlisting}
Но анализатор выдаст предупреждение, о совпадении веток \textit{if} и \textit{else}. Помимо 
ложного срабатывания при использовании пустых веток, часто анализатор выдает предупреждение для
кода:
\begin{lstlisting}
	if (err1)
		return -1;
	else if (err2)
		return -1;
	else
		return 0;
\end{lstlisting}
Приведенный код соответсвует следующему:
\begin{lstlisting}
	if (err1 || err2)
		return -1;
	else
		return 0;
\end{lstlisting}
Но для удобства читаемости исходного кода была использована конструкция \textit{if-else-if}. И так как
анализатор находит две одинаковые ветки (\textit{return -1;}), выдается предупреждение.

\subsection*{Ошибки второго рода}
Реализованный модуль для статических проверок построен на основе сопоставления участков кода заранее
заданным шаблонам, которые соответствуют фрагментам кода, потенциально содержащим ошибки.
Вследствие чего, эффективность нахождения ошибок реализованного модуля зависит от количества запрограммированных шаблонов.
Однако реализованные проверки могут находить не все ошибки. Рассмотрим пример:
\begin{lstlisting}
if (( (A || B) && C) || ((A && C) || (A && B)))
\end{lstlisting}
Как видно, левая часть \textbf{( (A || B) \&\& C)} эквивалентна правой части \textbf{((A \&\& C) || (A \&\& B))}
оператора \textbf{||}, но анализатор не выдаст никаких предупреждений так как абстрактное синтаксическое дерево 
для левой и правой части различны. Поэтому при использовании кода, которые содержит ошибку, но не соответсвует шаблону,
предупреждение не будет выданно.

%%%%%%%%%%%%%%%%%%%%%%%%%%
\subsection{Тестирование на реальных проектах}
%%%%%%%%%%%%%%%%%%%%%%%%%%
Для проведения тестирования на реальных проектах использовались проекты cURL\cite{curl}, MTR\cite{mtr} и irssi\cite{irssi}.
Приведенных проекты имеют открытый исходный код и написаны на языке C. 

Для компиляции проектов с разработанным модулем расширения использовался скрипт <<configure>>, который на основе 
заданных параметров генерирует <<Makefiles>>:
\begin{lstlisting}
./configure CC=/usr/local/bin/clang CFLAGS='-Xclang -load -Xclang <путь к модулю> -Xclang -add-plugin -Xclang CppSpotter'
\end{lstlisting}
Затем вызывалась команда <<make>>, которая начинала процесс компиляции.

\subsection*{Прмеры найденных ошибок}
Одной из выявленных ошибок в проекте MTR является следующий фрагмент кода:
\begin{lstlisting}
for (k=0; k < mpls->labels && enablempls; k++) {
	if((k+1 < mpls->labels) || (mpls->labels == 1)) {
    	/* if we have more labels */
        printw("\n    [MPLS: Lbl %lu Exp %u S %u TTL %u]", mpls->label[k], mpls->exp[k], mpls->s[k], mpls->ttl[k]);
    } else {
        /* bottom label */
        printw("\n    [MPLS: Lbl %lu Exp %u S %u TTL %u]", mpls->label[k], mpls->exp[k], mpls->s[k], mpls->ttl[k]);
    }
}
\end{lstlisting}
Как видно, ветки оператора <<if>> и <<else>> полностью совпадают и не понятно зачем тогда нужен оператор <<if>>. Скорее
всего здесь содержится ошибка.

Другим примером потенциальной ошибки является следующий код из проекта cURL:
\begin{lstlisting}
#define POLLOUT     0x04
...
#ifndef POLLWRNORM
#define POLLWRNORM POLLOUT
#endif
...
if(pfd[num].revents & (POLLWRNORM|POLLOUT))
\end{lstlisting}
Предупреждение выдается на выражение (POLLWRNORM|POLLOUT), так как слева и справа от битового оператора
<<|>> используется одно и тоже число. 

При использовании функции \textit{strlen} иногда необходимо подсчитать длину строки без первого
символа, для этого можно использовать конструкцию \textit{strlen(text+1)}, но с точки зрения анализатора,
данный участок кода содержит потенциальную ошибку и будет выдано предупреждение. Похожее ложное 
срабатывание происходит, когда необходимо выделить памяти для строки без учета символа '\\0'. 
Примером ложного срабатывания является код из проекта irssi:
\begin{lstlisting}
args = g_strdup(ptr);
ptr = strstr(args, " :");
if (ptr != NULL)
	g_memmove(ptr+1, ptr+2, strlen(ptr+1));
\end{lstlisting}
В приведенном коде выведется предупреждение на фрагмент кода {\em strlen(ptr+1)}.
Однако из текста программы видно, что использование в качестве аргумента {\em ptr+1} сделано
специально для пропуска пробела.   
    
\subsection*{Анализ скорости компиляции с разработанным модулем}
Для сравнения скорости компиляции производилось несколько сборок с использованием модуля расширения и без.
С помощью Unix-утилиты <<time>> производилось измерение времени компиляции. 
Результаты измерения времени представлены в таблицах \ref{table:curlTable}-\ref{table:irssiTable}.
\begin{table}[h!]
\caption{Время компиляции для проекта cURL (в секундах)}
\label{table:curlTable}
\centering
\begin{tabular}{| c | c | c |}
\hline
№ & Компиляция с модулем & Компиляция без модуля \\
\hline
1 & 340 & 306 \\
2 & 335 & 310 \\
3 & 338 & 304 \\
Среднее & 337 & 306 \\
\hline
\end{tabular}
\end{table}

\begin{table}[h!]
\caption{Время компиляции для проекта MTR (в секундах)}
\label{table:mtrTable}
\centering
\begin{tabular}{| c | c | c |}
\hline
№ & Компиляция с модулем & Компиляция без модуля \\
\hline
1 & 26 & 18 \\
2 & 21 & 19 \\
3 & 24 & 17 \\
Среднее & 24 & 18 \\
\hline
\end{tabular}
\end{table}

\begin{table}[h!]
\caption{Время компиляции для проекта irssi (в секундах)}
\label{table:irssiTable}
\centering
\begin{tabular}{| c | c | c |}
\hline
№ & Компиляция с модулем & Компиляция без модуля \\
\hline
1 & 343 & 291 \\
2 & 338 & 285 \\
3 & 340 & 280 \\
Среднее & 340 & 285 \\
\hline
\end{tabular}
\end{table}

%%%%%%%%%%%%%%%%%%%%%%%%%%
\section{Выводы}
%%%%%%%%%%%%%%%%%%%%%%%%%%
В результате проведенного тестирования были выявлены случаи, когда модуль статических проверок
выдает ложные предупреждения и в каких случаях не находит ошибки. Для улучшения точности срабатываний модуля 
необходимо произвести более тщательное тестирование проверок, которые чаще всего представляют ложные 
предупреждения. Также необходимо увеличить количество доступных проверок, чтобы модуль умел находить
как можно больше ошибок. Был произведен анализ затрачиваемого времения при компиляции с использованием 
модуля расширения и без него. Как видно из данного анализа, чем сложнее проект, тем больше
необходимо затратить времени на синтаксический анализ по сравнению с обычной компиляцией.
