%%%%%%%%%%%%%%%%%%%%%%%%%%%%%%%%%%%%%%%%%%%%%%%%%%%%%%%%%%%%%%%%%%%%%%%%%%%%%%%%
\chapter{Тестирование и анализ разработанного модуля}
%%%%%%%%%%%%%%%%%%%%%%%%%%%%%%%%%%%%%%%%%%%%%%%%%%%%%%%%%%%%%%%%%%%%%%%%%%%%%%%%

%%%%%%%%%%%%%%%%%%%%%%%%%%%%%%%%%%%%%%%%%%%%%%%%%%%%%%%%%%%%%%%%%%%%%%%%%%%%%%%%
\section{Тестирование}
%%%%%%%%%%%%%%%%%%%%%%%%%%%%%%%%%%%%%%%%%%%%%%%%%%%%%%%%%%%%%%%%%%%%%%%%%%%%%%%%
Показать нахождение ошибок. Показать в каких случаях происходит ложное срабатывание и когда
наоборот не происходит нахождения ошибки.

%%%*++scan == *++match && *++scan == *++match

Диагностическое сообщение не выдается, если сравниваются два идентичных выражения типа float или double. 
Такое сравнение позволяет определить, является ли значение NaN. Пример кода, реализующего подобную проверку:
bool isnan(double X) { return X != X; }

Наличие двух пустых веток считается корректной и безопасной ситуацией. Подробные конструкции 
можно часто встретить при использовании макросов. Пример безопасного кода:
if (exp) {
} else {
}
%%%%%%%%%%%%%%%%%%%%%%%%%%%%%%%%%%%%%%%%%%%%%%%%%%%%%%%%%%%%%%%%%%%%%%%%%%%%%%%%
\section{Анализ}
%%%%%%%%%%%%%%%%%%%%%%%%%%%%%%%%%%%%%%%%%%%%%%%%%%%%%%%%%%%%%%%%%%%%%%%%%%%%%%%%

Тут будет сравнение скорости компиляции с плагином, без него и с частично отключенными проверками.